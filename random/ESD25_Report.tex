\documentclass[sigconf]{acmart}

%% Rights information
\setcopyright{acmcopyright}
\copyrightyear{2025}
\acmYear{2025}
\acmConference{ESD811S}{2025}{Namibia}

\usepackage{booktabs}
\usepackage{graphicx}
\usepackage{url}
\usepackage{hyperref}

%% Set the title and author details
\title{\textbf{Enhancing Tourism Through Digitalization: Implementation of Smart Booking System for Pulau Untung Jawa Island}}

\author{Tashinga Ryan Manunure}
\email{tashiemanunure@gmail.com}
\orcid{1234-5678-9012}
\affiliation{
  \institution{Namibia University of Science and Technology}
  \city{Windhoek}
  \country{Namibia}
}
\author{Maria Tulina Matheus}
\email{azeamatheus@gmail.com}
\orcid{1234-5678-9012}
\affiliation{
  \institution{Namibia University of Science and Technology}
  \city{Windhoek}
  \country{Namibia}
}

%% CCS Concepts
\begin{CCSXML}
<ccs2012>
<concept>
<concept_id>10002951.10003227.10003351</concept_id>
<concept_desc>Information systems~Web applications</concept_desc>
<concept_significance>500</concept_significance>
</concept>
<concept>
<concept_id>10010147.10010257.10010293.10010294</concept_id>
<concept_desc>Computing methodologies~Artificial intelligence</concept_desc>
<concept_significance>300</concept_significance>
</concept>
<concept>
<concept_id>10002951.10003227.10003241.10003244</concept_id>
<concept_desc>Information systems~Database management</concept_desc>
<concept_significance>300</concept_significance>
</concept>
</ccs2012>
\end{CCSXML}

\ccsdesc[500]{Information systems~Web applications}
\ccsdesc[300]{Computing methodologies~Artificial intelligence}
\ccsdesc[300]{Information systems~Database management}

%% Abstract
\begin{abstract}
This paper presents the development and implementation of a digital tourism solution for Pulau Untung Jawa Island, focusing on the integration of emerging technologies in the tourism sector. We describe the implementation of a smart homestay booking system that incorporates artificial intelligence features to enhance the tourist experience. The system addresses key challenges in island tourism through digital transformation, including accommodation management, tourist information accessibility, and local business integration. Our solution demonstrates how emerging technologies can be leveraged to promote sustainable tourism development in island communities while preserving local cultural heritage. This project was developed as part of the Global Intercultural Project Experience (GIPE++) program, demonstrating international academic collaboration in addressing real-world tourism challenges.
\end{abstract}

%% Keywords
\keywords{Tourism Digitalization, Smart Tourism, Homestay Management, Artificial Intelligence, Island Tourism, Sustainable Development, Digital Transformation, GIPE++}

\begin{document}

\maketitle

\section{Introduction}

The digital transformation of tourism services has become increasingly crucial for developing sustainable and accessible tourist destinations. Pulau Untung Jawa, an island destination in Indonesia, presents unique opportunities and challenges in implementing digital solutions for tourism management. This paper discusses the development and implementation of a comprehensive digital platform designed to enhance the tourist experience while supporting local business growth through technology integration, developed as part of the Global Intercultural Project Experience (GIPE++) program.

Island tourism faces distinct challenges including limited accessibility to accommodation information, inefficient booking and payment systems, lack of standardized quality control for homestays, and insufficient digital presence for local businesses. We argue that the implementation of emerging technologies, particularly artificial intelligence and modern web technologies, can address these challenges while promoting sustainable tourism development.

The aim of this project is to develop a comprehensive digital platform that enhances the tourism experience for Pulau Untung Jawa Island through the integration of smart booking systems, AI-powered chatbots, and modern web technologies. Our approach demonstrates how emerging technologies can be effectively leveraged to bridge the digital divide in island tourism while preserving cultural heritage and supporting local economic development. This work represents a collaborative effort between Namibian and international partners through the GIPE++ Digital Platform stream, showcasing how academic partnerships can produce practical solutions with real-world impact.

\section{Related Work/Literature Review}

The digitalization of heritage tourism has gained significant attention in recent academic literature, with researchers exploring various technological approaches to enhance destination choice and visitor experience. Khoa and Huynh (2024) conducted a comprehensive study on improving destination choice in heritage tourism through tourism digitalization, using Vietnam as a case study \cite{khoa2024heritage}. Their research demonstrates that digital platforms significantly influence tourist decision-making processes and destination selection, particularly in heritage tourism contexts where cultural authenticity and information accessibility are paramount. The study revealed that tourism digitalization, encompassing accessibility and interactivity, significantly influenced destination image, positively affecting destination choice, while informativeness and personalization showed less impact.

The concept of smart tourism has evolved considerably, with Gretzel et al. (2015) defining it as tourism supported by integrated efforts at a destination to collect, aggregate, and harness data derived from physical infrastructure, social connections, government/organizational sources, and human bodies/minds in combination with the use of advanced technologies to transform that data into on-site experiences and business value propositions \cite{gretzel2015smart}. This foundational work has influenced subsequent research in digital tourism platforms, establishing the theoretical framework for technology integration in tourism services.

Recent studies have emphasized the role of digital technologies in transforming heritage tourism experiences. Buonincontri and Micera (2016) explored experience co-creation in smart tourism destinations through multiple case analyses of European destinations, highlighting how digital technologies facilitate collaborative tourism experiences \cite{buonincontri2016experience}. Their findings suggest that smart tourism technologies enable more personalized and engaging visitor experiences while supporting destination management objectives.

The integration of mobile technologies and interactive platforms in tourism has been extensively studied by Jeong and Shin (2020), who demonstrated that tourists' experiences with smart tourism technology at smart destinations significantly influence their behavioral intentions \cite{jeong2020tourists}. Their research indicates that tourists increasingly rely on digital platforms for accommodation booking and destination information, making digital presence essential for tourism businesses, particularly in heritage tourism contexts.

Furthermore, the accessibility aspect of digital tourism has been addressed by Tseng et al. (2015), who examined how travel blogs function as destination image formation agents, arguing that accessible digital content significantly influences tourist perceptions and decision-making processes \cite{tseng2015travel}. This perspective aligns with our approach to developing a platform that not only enhances tourist experience but also supports local economic development in Pulau Untung Jawa through improved information accessibility and interactive features.

The application of artificial intelligence in tourism digitalization has been explored by Huang et al. (2017), who investigated smart tourism technologies in travel planning, emphasizing the role of exploration and exploitation in technology adoption \cite{huang2017smart}. Their findings suggest that AI-powered systems can significantly improve customer satisfaction and operational efficiency in tourism businesses, particularly in resource-constrained environments such as island destinations.

\section{Process Description}

Our development process followed an agile methodology, incorporating user-centered design principles and emerging technology integration as part of the GIPE++ Digital Platform stream requirements. We maintain that this approach was essential for creating a solution that addresses real-world challenges while leveraging cutting-edge technologies and fostering international collaboration.

\subsection{Technology Stack Selection}

The backend development utilized Node.js with TypeScript, providing type safety and modern JavaScript features essential for scalable web applications. We implemented Express.js as the web framework, chosen for its flexibility, extensive middleware ecosystem, and robust routing capabilities. The backend architecture follows RESTful API principles with comprehensive error handling and input validation using express-validator.

For data persistence, we integrated PostgreSQL database hosted on Supabase infrastructure, utilizing direct PostgreSQL connections through the \texttt{pg} library rather than Supabase's client SDK. This approach provides greater control over database operations while maintaining the benefits of managed hosting. The database configuration includes connection pooling with SSL support for secure data transmission.

The frontend development employed React.js with modern functional components and hooks, ensuring a responsive and interactive user interface. We integrated contemporary styling approaches for rapid UI development with consistent design patterns optimized for mobile and desktop experiences.

\subsection{AI Integration - Google Gemini Implementation}

The AI chatbot implementation represents the core emerging technology component of our system, utilizing Google's Gemini 2.0 Flash model rather than traditional chatbot frameworks. We postulate that this choice provides superior natural language understanding and contextual awareness compared to rule-based alternatives, aligning with the GIPE++ program's emphasis on innovative technology adoption.

The chatbot system, named "Pulau Pal," incorporates several advanced features:

\begin{itemize}
    \item \textbf{Intelligent Query Classification:} Distinguishes between tourism-related and general knowledge queries
    \item \textbf{Context-Aware Responses:} Maintains conversation history and understands current page context
    \item \textbf{Dynamic Knowledge Integration:} Real-time access to homestay availability and booking data
    \item \textbf{Multi-Modal Interface:} Three interaction modes - Chat, Discover, and Plan Trip tabs
    \item \textbf{Fallback Mechanisms:} Graceful degradation when API services are unavailable
\end{itemize}

The implementation includes sophisticated prompt engineering with system-level instructions that define the chatbot's personality as a tourism assistant while maintaining safety guidelines and content filtering through Gemini's built-in safety settings.

\subsection{Database Design and Implementation}

We designed a comprehensive relational database schema supporting multiple user types, accommodation management, and booking operations. The schema includes eleven interconnected tables:

\begin{itemize}
    \item \textbf{User Management:} \texttt{admin\_users} and \texttt{landing\_page\_user} tables with role-based access control
    \item \textbf{Accommodation System:} \texttt{homestay}, \texttt{homestayRoom}, and \texttt{homestayImages} tables supporting detailed property management
    \item \textbf{Feature Management:} \texttt{room\_features} and \texttt{relation\_feature\_room} tables enabling flexible amenity assignment
    \item \textbf{Booking Operations:} \texttt{booking} table with comprehensive reservation tracking
    \item \textbf{Business Logic:} \texttt{reviews}, \texttt{payments}, and \texttt{notifications} tables supporting complete business workflows
\end{itemize}

The database utilizes PostgreSQL enums for status fields, ensuring data integrity and consistent state management across the application.

\subsection{API Development and Security}

RESTful API endpoints were developed following industry best practices, including proper HTTP status codes, consistent response formats, and comprehensive error handling. The API architecture includes:

\begin{itemize}
    \item \textbf{Authentication:} JWT-based authentication with bcrypt password hashing
    \item \textbf{Authorization:} Role-based access control for different user types
    \item \textbf{Validation:} Input validation using express-validator for all endpoints
    \item \textbf{Security:} CORS configuration, SQL injection prevention, and rate limiting considerations
    \item \textbf{Email Integration:} Nodemailer integration for booking confirmations and notifications
\end{itemize}

\section{Description of End Result/Prototype}

The resulting platform consists of a comprehensive digital ecosystem for Pulau Untung Jawa tourism management, developed through the GIPE++ Digital Platform stream collaboration. We postulate that this system demonstrates effective integration of emerging technologies in addressing real-world tourism challenges while showcasing international academic cooperation.

\subsection{User Stories and System Functionality}

\textbf{Tourist User Stories:}
\begin{itemize}
    \item As a tourist, I want to browse available homestays with detailed room information and high-quality images
    \item As a tourist, I want to interact with "Pulau Pal" AI chatbot to get personalized island recommendations
    \item As a tourist, I want to book specific rooms with real-time availability checking and secure payment processing
    \item As a tourist, I want to receive email confirmations and manage my booking status
    \item As a tourist, I want to leave reviews and ratings for my accommodation experience
\end{itemize}

\textbf{Homestay Owner User Stories:}
\begin{itemize}
    \item As a homestay owner, I want to manage multiple room types with individual pricing and features
    \item As a homestay owner, I want to upload and organize property images with primary image designation
    \item As a homestay owner, I want to receive and manage booking requests with automated notifications
    \item As a homestay owner, I want to track payment status and booking history through an admin dashboard
    \item As a homestay owner, I want to respond to customer reviews and maintain property ratings
\end{itemize}

\subsection{Technical Architecture}

The system architecture follows a modern three-tier model with microservices considerations:

\textbf{Presentation Layer:} React.js frontend with responsive design, featuring the Gemini-powered chatbot interface with island-themed UI elements including mystical animations and gradient backgrounds. The frontend is optimized for deployment on Vercel platform for global accessibility.

\textbf{Application Layer:} Node.js/Express.js backend providing RESTful APIs, JWT authentication services, and AI chatbot integration. The backend includes comprehensive middleware for error handling, CORS management, and request validation. The system is designed for deployment on Railway platform with environment-based configuration.

\textbf{Data Layer:} PostgreSQL database hosted on Supabase infrastructure, providing ACID compliance, real-time capabilities, and automated backups. The database design supports complex queries for availability checking and booking management.

\subsection{AI Chatbot Implementation Details}

The "Pulau Pal" chatbot represents the primary emerging technology component, implemented with sophisticated features:

\begin{itemize}
    \item \textbf{Google Gemini 2.0 Flash Integration:} Advanced natural language processing with configurable parameters (temperature, max tokens)
    \item \textbf{Context Management System:} Page scanner service extracts current page content for contextual responses
    \item \textbf{Knowledge Base Integration:} Static island information combined with dynamic homestay data from the booking system
    \item \textbf{Multi-Language Support:} English and Indonesian language capabilities for international and local tourists
    \item \textbf{Safety and Content Filtering:} Gemini API safety settings prevent harmful content generation
    \item \textbf{Fallback Response System:} Pre-defined tourism-focused responses when AI services are unavailable
\end{itemize}

The chatbot interface features three distinct interaction modes: Chat for general conversation, Discover for island exploration, and Plan Trip for travel planning assistance.

\subsection{Key System Features}

\begin{itemize}
    \item \textbf{Smart Room Search:} AI-powered search with natural language queries and availability filtering
    \item \textbf{Real-time Booking System:} Instant availability checking with conflict prevention and booking confirmation
    \item \textbf{Comprehensive Payment Integration:} Secure payment processing with multiple payment method support
    \item \textbf{Advanced Review System:} Rating mechanisms with moderation capabilities
    \item \textbf{Admin Dashboard:} Complete property management interface with analytics and reporting
    \item \textbf{Email Automation:} Automated booking confirmations and status updates using Nodemailer
    \item \textbf{Image Management:} High-quality image upload and optimization with primary image designation
    \item \textbf{Mobile-First Design:} Responsive interface optimized for all device types
\end{itemize}

\section{Reflection}

\subsection{Challenges Experienced}

During the development process, we encountered several significant challenges that provided valuable learning experiences within the GIPE++ collaborative framework. The integration of Google Gemini AI presented initial complexity in managing API costs and response optimization. We believe that careful prompt engineering and response caching were crucial for creating an efficient and cost-effective AI implementation. The challenge of maintaining context across conversation turns required sophisticated state management and conversation history tracking.

Database design and migration challenges emerged when working with PostgreSQL on Supabase, particularly in managing concurrent booking requests and preventing double-bookings. We maintain that implementing proper transaction management and optimistic locking mechanisms was essential for data consistency. The complexity of the room-feature relationship system required careful normalization to support flexible amenity assignment while maintaining query performance.

Cross-origin resource sharing (CORS) configuration presented deployment challenges when integrating the React frontend and Node.js backend across different hosting platforms (Vercel and Railway). The solution required careful configuration of environment variables and security policies to support both development and production environments.

Email integration using Nodemailer required extensive testing across different email providers to ensure reliable delivery of booking confirmations and notifications. We argue that implementing proper error handling and retry mechanisms was essential for maintaining user trust in the booking system.

The international collaboration aspect of GIPE++ also presented coordination challenges, requiring effective communication across different time zones and cultural contexts while maintaining project momentum and technical consistency.

\subsection{Future Work}

We argue that several enhancements could further improve the platform's effectiveness and extend the GIPE++ collaboration impact:

\begin{itemize}
    \item \textbf{Advanced Machine Learning:} Implementation of recommendation algorithms based on user behavior patterns and booking history
    \item \textbf{IoT Integration:} Smart home features for homestays including automated check-in systems and environmental monitoring
    \item \textbf{Blockchain Payment Systems:} Secure and transparent payment processing using cryptocurrency for international transactions
    \item \textbf{Augmented Reality Features:} Virtual tours and AR-enhanced local information for immersive destination exploration
    \item \textbf{Predictive Analytics:} Demand forecasting and dynamic pricing optimization based on seasonal patterns and local events
    \item \textbf{Voice Interface Integration:} Voice-activated chatbot interactions for hands-free tourism assistance
    \item \textbf{Real-time Translation:} Multi-language support expansion with real-time translation capabilities
    \item \textbf{GIPE++ Network Expansion:} Integration with other GIPE++ Digital Platform projects for cross-destination tourism experiences
\end{itemize}

\subsection{Possible Contexts of Applications}

The developed platform demonstrates broader applicability beyond Pulau Untung Jawa, particularly within the GIPE++ network and similar international collaboration contexts. We postulate that similar implementations could benefit:

\begin{itemize}
    \item \textbf{Indonesian Archipelago:} Other Indonesian islands seeking digital transformation in tourism management
    \item \textbf{Developing Tourism Markets:} Small-scale tourism destinations in developing countries requiring affordable digital solutions
    \item \textbf{Heritage Tourism Sites:} Cultural and historical destinations requiring digital preservation and promotion
    \item \textbf{Community-Based Tourism:} Local community initiatives seeking to integrate technology while preserving cultural authenticity
    \item \textbf{Sustainable Tourism Projects:} Eco-tourism destinations requiring visitor management and environmental impact monitoring
    \item \textbf{Rural Tourism Development:} Remote destinations needing digital connectivity to global tourism markets
    \item \textbf{GIPE++ Partner Destinations:} Other locations within the GIPE++ network seeking similar digital transformation solutions
\end{itemize}

The integration of AI technologies in tourism platforms represents a growing trend that could revolutionize how tourists interact with destinations and how local businesses manage their operations. We believe that this project demonstrates the potential for emerging technologies to create meaningful impact in tourism development while supporting local economic growth and cultural preservation through international academic collaboration.

\section{Conclusion}

This project successfully demonstrates the implementation of emerging technologies in heritage tourism digitalization, specifically addressing the challenges faced by Pulau Untung Jawa Island through the Global Intercultural Project Experience (GIPE++) Digital Platform stream. The integration of Google Gemini AI-powered chatbots, modern web technologies, and comprehensive booking systems creates a robust platform that enhances tourist experience while supporting local business development.

We maintain that the success of this implementation validates the potential for digital transformation in island tourism contexts through international academic collaboration. The platform not only addresses immediate operational challenges but also establishes a foundation for future technological enhancements and sustainable tourism development. The sophisticated use of Google's Gemini AI for contextual tourism assistance represents a significant advancement in applying large language models to domain-specific applications within the heritage tourism sector.

The project's alignment with the Global Intercultural Project Experience (GIPE++) objectives demonstrates how academic collaboration between Namibian and international partners can produce practical solutions with real-world impact. We argue that this approach to technology integration in tourism could serve as a model for similar destinations seeking digital transformation while preserving cultural heritage and promoting sustainable development through the GIPE++ network.

The technical architecture combining Node.js/TypeScript backend, React frontend, PostgreSQL database, and AI integration provides a scalable foundation that can be adapted to various tourism contexts within and beyond the GIPE++ framework. We believe that the lessons learned from this implementation will contribute to the broader understanding of emerging technology applications in heritage tourism digitalization and serve as a reference for future GIPE++ Digital Platform stream projects.

\bibliographystyle{ACM-Reference-Format}
\begin{thebibliography}{10}

\bibitem{khoa2024heritage}
Khoa, B.T. and Huynh, T.T. 2024. How to improve the destination choice in heritage tourism through tourism digitalization: Case of Vietnam. In \textit{2024 2nd International Conference on Sustaining Heritage: Embracing Technological Advancements (ICSH)}. IEEE, 26-29. DOI: 10.1109/ICSH62408.2024.10779832

\bibitem{gretzel2015smart}
Gretzel, U., Sigala, M., Xiang, Z., and Koo, C. 2015. Smart tourism: foundations and developments. \textit{Electronic Markets} 25, 3 (2015), 179-188.

\bibitem{buonincontri2016experience}
Buonincontri, P. and Micera, R. 2016. The experience co-creation in smart tourism destinations: a multiple case analysis of European destinations. \textit{Information Technology \& Tourism} 16, 4 (2016), 285-315.

\bibitem{jeong2020tourists}
Jeong, M. and Shin, H.H. 2020. Tourists' experiences with smart tourism technology at smart destinations and their behavior intentions. \textit{Journal of Travel Research} 59, 8 (2020), 1464-1477.

\bibitem{tseng2015travel}
Tseng, C., Wu, B., Morrison, A.M., Zhang, J., and Chen, Y.-c. 2015. Travel blogs on China as a destination image formation agent: A qualitative analysis using Leximancer. \textit{Tourism Management} 46 (2015), 347-358.

\bibitem{huang2017smart}
Huang, C.D., Goo, J., Nam, K., and Yoo, C.W. 2017. Smart tourism technologies in travel planning: The role of exploration and exploitation. \textit{Information \& Management} 54, 6 (2017), 757-770.

\end{thebibliography}

\end{document}